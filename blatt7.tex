\documentclass[fleqn,12pt]{scrartcl}
\usepackage[utf8]{inputenc}
\usepackage{color}
\usepackage[ngerman]{babel}
\usepackage{amssymb}
\usepackage{amsthm}
\usepackage{amsmath}
\usepackage{hyperref}
\usepackage{wasysym}
\usepackage{scrpage2}
\usepackage{tikz}
\usetikzlibrary{intersections}
\pagestyle{scrheadings}
\clearscrheadfoot
\ohead{\pagemark}
\ihead{Magnus Berendes, 21752155}
\ifoot{\today}
\ofoot{\blattn}
%\setheadtopline{1pt}
\setheadsepline{0.4pt}
\setfootsepline{0.4pt}
\usepackage{enumitem}
\setenumerate[0]{label=\alph*)}
\newcommand{\id}{\, \mathrm{d}}
\newcommand{\intl}{\int\displaylimits}
% New definition of square root:
% it renames \sqrt as \oldsqrt
\let\oldsqrt\sqrt
% it defines the new \sqrt in terms of the old one
\def\sqrt{\mathpalette\DHLhksqrt}
\def\DHLhksqrt#1#2{%
	\setbox0=\hbox{$#1\oldsqrt{#2\,}$}\dimen0=\ht0
	\advance\dimen0-0.2\ht0
	\setbox2=\hbox{\vrule height\ht0 depth -\dimen0}%
{\box0\lower0.4pt\box2}}

\newcommand{\karos}[2]{
	\begin{tikzpicture}
		\draw[step=0.5cm, color=gray] (0,0) grid (#1 cm , #2 cm);
	\end{tikzpicture}
}
\newcommand{\abs}[1]{
	\left \vert #1 \right \vert
}
\newcommand{\absbb}[1]{
	\left \Vert #1 \right \Vert
}

\newcommand{\blattn}{Blatt 7}
\begin{document}
\section*{\blattn}
Bearbeitet/Zur Korrektur:

\noindent
\begin{Form}
	\CheckBox{Aufgabe A19}\\
	\CheckBox{Aufgabe A20}\\
	\CheckBox{Aufgabe A21}\\
	%\CheckBox{Aufgabe A4}
\end{Form}

\section{A19}
$x_{1,2}$ Holz, bzw. Wolle in kg
\begin{align*}
	\text{Zielfunktion: } g(x) = 48 x_1 + 9 x_2\\
	\text{NB: }  
	6x_1 + x_2 &\leq 180 \\
						  80 x_1 + 20 x_2 &\leq 3200\\
\end{align*}
\begin{align*}
	&\min\, (-48 x_1 - 9x_2)\\
 &\begin{array}{lllllllll}
	6&x_1 &+ &&x_2 &+ &x_3 & \quad &= 180\\
	80&x_1 &+ &20&x_2 &+ &  &x_4 &= 3200\\
\end{array}\\
	&x_1, x_2, x_3, x_4 \geq 0
\end{align*}
Basis: $\{3, 4\}$, $x_1, x_2 \overset != 0$\\
$\Rightarrow x_3 = 180, x_4 = 3200$\\
$\Rightarrow x > 0$, Basisl"osung zul"assig
\begin{align*}
	A = \begin{pmatrix}
		6 & 1 & 1 & 0\\
		80 & 20 & 0 & 1
	\end{pmatrix}, b = \begin{pmatrix}
	180\\ 3200 \end{pmatrix}, c^T = \begin{pmatrix} -48 & -9 & 0 & 0 \end{pmatrix}
\end{align*}
\begin{align*}
	T(B) &= 
	\begin{array}{cccc|c}
		-48 & -9 & 0 & 0 & 0\\
		\hline
		6 & 1 & 1 & 0 & 180\\
		80 & 20 & 0 & 1 & 3200
	\end{array}\\
	&\\
	&180/6 = 30, 3200/80 = 40\\
	B' &= \{1, 4\}\\
	I &\rightarrow I + 8II\\
	II &\rightarrow \frac16 II\\
	III &\rightarrow III - \frac{80}6 II\\
	T(B') &=
	\begin{array}{cccc|c}
		0 & -1 & 8 & 0 & 1440\\
		\hline
		1 & \frac16 & \frac16 & 0 & 30\\
		0 & \frac{20}3 & \frac{40}3 & 1 & 800
	\end{array}
	\\
	\\
	30*6 &= 180, 800*\frac3{20}= 120\\
	B'' &= \{1, 2\}\\
	I &\rightarrow I + \frac3{20}III\\
	II &\rightarrow II - \frac1{40}III\\
	III &\rightarrow \frac3{20} III\\
	T(B'') &= \begin{array}{cccc|c}
		0 & 0 & 10 & \frac3{20} & 1560 \\
		1 & 0 & -\frac16 & -\frac1{40} & 10 \\
		0 & 1 & 2 & \frac3{20} & 120
	\end{array}
\end{align*}
Abbruchbedingung erreicht! $x_1 = 10, x_2 = 120$\\
Es werden 10 kg Holz und 120 kg Wolle verarbeitet, der Gewinn betr"agt 1560. \\
(Die Nebenbedingung, dass nur 20kg Holz zur Verf"ugung stehen wurde zuerst "ubersehen. Da diese L"osung die Nebenbedingung jedoch erf"ullt, gilt die L"osung auch unter Einbeziehung der Nebenbedingung.)
\section{A20}
\begin{enumerate}
	\item
Klee zu kaufen w"are eher d"amlich, da Karotten mehr Vitamine beinhalten und genausoviele Mineralstoffe, dabei aber 3 Euro billiger sind, er sollte also lieber keinen Klee kaufen.
\item
	Die Zielfunktion sind die Kosten, diese sollen minimiert werden: $\min\, 30x_1 + 18 x_2$, $x_1$: Kraftfutter in P, $x_2$: Karotten in P
	\begin{align*}
		\min &\,30x_1 + 18 x_2\\
		\text{NB: }& x_1 + 2x_2 \geq 8\\
							 & 2x_1 + x_2 \geq 7\\
				 &x_1, x_2 \geq 0
	\end{align*}
	In Standardform:
	\begin{align*}
		\max &\,-30x_1 - 18 x_2 \\
		\text{s.t.}& \\
												 &\begin{array}{ccccccccc}
		-&x_1 &- &2x_2 &+ &x_3&& &=-8\\
		-&2x_1 &- &x_2 & &&+ &x_4 &= -7\\
	\end{array}\\
	&x_1, x_2, x_3, x_4 \geq 0
	\end{align*}
	(F"ur dualen Simplex ist max-Problem nötig)

	Basis = $\{3, 4\},\, x_1, x_2 \overset!= 0$
	\begin{align*}
		\Rightarrow x_3 = 8, x_4 = 7\\
		A=\begin{pmatrix}
			-1 & -2 & 1 & 0\\
			-2 & -1 & 0 & 1\\
		\end{pmatrix}, b= \begin{pmatrix}
		-8\\-7\end{pmatrix}, c^T = \begin{pmatrix} 30 & 18 & 0 & 0 \end{pmatrix}
	\end{align*}
	\begin{align*}
		T(B) = \begin{array}{cccc|c}
			-30 & -18 & 0 & 0 & 0\\
			\hline
			-1 & -2 & 1 & 0 & -8\\
			-2 & -1 & 0 & 1 & -7
		\end{array}
	\end{align*}
	\begin{align*}
		(\frac{-18}{-2} &= 9,\,\quad
		\frac{-30}{-1} = 30\\)
															&\Rightarrow B'=\{2,4\}\\
		I &\rightarrow I - 9II\\
		II &\rightarrow -\frac12 II\\
		III &\rightarrow III - \frac12 II
	\end{align*}
	\begin{align*}
		T(B') = \begin{array}{cccc|c}
			-21 & 0 & -9 & 0 & 72\\
			\hline
			\frac12 & 1 & -\frac12 & 0 & 4\\
			-\frac32 & 0 & -\frac12 & 1 & -3
		\end{array}
	\end{align*}
	\begin{align*}
		(\frac{-9}{-\frac12} &= 18, \quad \frac{-21}{-\frac32} = 14)\\
		&\Rightarrow B'' = \{1, 2\}\\
		I &\rightarrow I + 14III\\
		II &\rightarrow II + \frac13 III\\
		III &\rightarrow -\frac23 III\\
	\end{align*}
	\begin{align*}
		T(B'') = \begin{array}{cccc|c}
			0 & 0 & -2 & -14 & 114\\
			\hline
			0 & 1 & -\frac13 & \frac13 & 3\\
			1 & 0 & \frac13 & -\frac13 & 2
		\end{array}
		\\
		\Rightarrow \text{Optimallösung!}
	\end{align*}

\item
	$x_1 = 2, x_2 = 3$\\
	Der Bauer kauft 2P Kraftfutter und 3P Karotten, das kostet ihn 114 Euro
\end{enumerate}
\section{A21}
$\min \langle c, x \rangle, \, Ax = b, \, x \geq 0$
\begin{enumerate}
	\item
	 LP existiert nicht
 \item
 $A= \begin{pmatrix} -1 & -1 \end{pmatrix}, \, b = 3, \, c = \begin{pmatrix} 1 & 1 \end{pmatrix}$
 \item
 $A= \begin{pmatrix} 1 & -1 \end{pmatrix}, \, b = 3, \, c = \begin{pmatrix} 1 & 1 \end{pmatrix}$
 \item
 $A= \begin{pmatrix} 1 & 1 \end{pmatrix}, \, b = 3, \, c = \begin{pmatrix} 1 & 1 \end{pmatrix}$
 \item
	 LP existiert nicht
 \item
 $A= \begin{pmatrix} 1 & 1 \end{pmatrix}, \, b = 3, \, c = \begin{pmatrix} -1 & 0 \end{pmatrix}$
 \item
 $A= \begin{pmatrix} 1 & 2 \end{pmatrix}, \, b = 1, \, c = \begin{pmatrix} 0 & 2 \end{pmatrix}$
\end{enumerate}


\end{document}
