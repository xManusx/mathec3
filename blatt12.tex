\documentclass[fleqn,12pt]{scrartcl}
\usepackage[utf8]{inputenc}
\usepackage{color}
\usepackage[ngerman]{babel}
\usepackage{amssymb}
\usepackage{amsthm}
\usepackage{amsmath}
\usepackage{gauss}
\usepackage{braket}
\usepackage{hyperref}
\usepackage{wasysym}
\usepackage{scrpage2}
\usepackage{tikz}
\usetikzlibrary{intersections}
\pagestyle{scrheadings}
\clearscrheadfoot
\ohead{\pagemark}
\ihead{Magnus Berendes, matrikelnr}
\ifoot{\today}
\ofoot{\blattn}
%\setheadtopline{1pt}
\setheadsepline{0.4pt}
\setfootsepline{0.4pt}
\usepackage{enumitem}
\setenumerate[0]{label=\alph*)}
\newcommand{\id}{\, \mathrm{d}}
\newcommand{\intl}{\int\displaylimits}
% New definition of square root:
% it renames \sqrt as \oldsqrt
\let\oldsqrt\sqrt
% it defines the new \sqrt in terms of the old one
\def\sqrt{\mathpalette\DHLhksqrt}
\def\DHLhksqrt#1#2{%
	\setbox0=\hbox{$#1\oldsqrt{#2\,}$}\dimen0=\ht0
	\advance\dimen0-0.2\ht0
	\setbox2=\hbox{\vrule height\ht0 depth -\dimen0}%
{\box0\lower0.4pt\box2}}

\newcommand{\karos}[2]{
	\begin{tikzpicture}
		\draw[step=0.5cm, color=gray] (0,0) grid (#1 cm , #2 cm);
	\end{tikzpicture}
}
\newcommand{\abs}[1]{
	\left \vert #1 \right \vert
}
\newcommand{\absbb}[1]{
	\left \Vert #1 \right \Vert
}
\newcommand{\BAR}{%
	\hspace{-\arraycolsep}%
	\strut\vrule % the `\vrule` is as high and deep as a strut
	\hspace{-\arraycolsep}%
}

%TODO
\newcommand{\blattn}{Blatt 12}
\begin{document}
\section*{\blattn}
Bearbeitet/Zur Korrektur:

\noindent
\begin{Form}
	\CheckBox{Aufgabe A34}\\
	\CheckBox{Aufgabe A35}\\
	\CheckBox{Aufgabe A36}\\
	%\CheckBox{Aufgabe A4}
\end{Form}

\section*{A34}
\begin{align*}
y' = \begin{pmatrix} 1 & 1 \\ -1 & 1 \end{pmatrix}y + \begin{pmatrix} \sin(t)\\ \cos(t) \end{pmatrix}
\end{align*}

\begin{itemize}
	\item
Homogene L"osung:
\begin{align*}
y' = \begin{pmatrix} 1 & 1 \\ -1 & 1 \end{pmatrix}y
\end{align*}
EW:
\begin{align*}
\det\begin{pmatrix}1-\lambda & 1 \\ -1 & 1-\lambda\end{pmatrix} = (1-\lambda)^2 + 1 \overset!=& 0\\
	(1-\lambda)^2 =& -1\\
	\pm(1-\lambda) =& i\\
	\Rightarrow& \lambda_1 = 1-i, \lambda_2 = 1+i\\
\end{align*}
EV:
\begin{align*}
	\lambda=1-i:&\\
							&\begin{gmatrix}[p]
		i & 1  &\BAR&0\\
		-1 & i  &\BAR& 0
		\rowops
		\add[-i]{0}{1}
	\end{gmatrix} \Rightarrow \begin{gmatrix}[p]
		i & 1  &\BAR&0 \\
		0 & 0  &\BAR&0
	\end{gmatrix}\\
	&\Rightarrow v_2 + iv_1 = 0\\
	&\Rightarrow v = \begin{pmatrix}i \\ 1 \end{pmatrix}\\
	\lambda = 1+i:&\\
														 &\begin{gmatrix}[p]
	-i & 1 \\
	-1 & -i
	\rowops
\add[i]{0}{1} 
\end{gmatrix} \Rightarrow \begin{pmatrix}-i & 1\\ 0 & 0 \end{pmatrix}\\
																						&\Rightarrow v_2 -iv_1 = 0\\
											&\Rightarrow
v_2 = \begin{pmatrix} -i\\1\end{pmatrix}
\end{align*}
\begin{align*}
y_H(t) = c_1\begin{pmatrix}i\\1\end{pmatrix}e^{(1-i)t} + c_2\begin{pmatrix}-i\\1\end{pmatrix}e^{(1+i)t}
\end{align*}

\item
Spezielle L"osung
\begin{align*}
	y_sp &= W(t) \cdot \int W^{-1}(t)b(t) \id t\\
	W &= \begin{pmatrix} 
		ie^{(1-i)t} & -ie^{(1+i)t}\\
e^{(1-i)t} & e^{(1+i)t}
\end{pmatrix}\\
W^{-i} &= \frac1{ie^{(1-i)t+(1+i)t} + ie^{(1+i)t+(1-i)t}}\begin{pmatrix}
e^{(1+i)t} & ie^{(1+i)t}\\
-e^{(1-i)t} & ie^{(1-i)t} \end{pmatrix}\\
&= \frac{1}{2ie^{2t}}
\begin{pmatrix}
e^{(1+i)t} & ie^{(1+i)t}\\
-e^{(1-i)t} & ie^{(1-i)t} \end{pmatrix}\\
										 &= \begin{pmatrix}
	\frac1{2i} e^{(-1+i)t} & \frac12 e^{(-1+i)t}\\
	-\frac{1}{2i} e^{(-1-i)t} & -\frac12 e^{(-1-i)t}\\
\end{pmatrix}\\
b(t) &= \begin{pmatrix}\sin(t)\\\cos(t)\end{pmatrix}
\end{align*}
\begin{align*}
	y_{SP} &= 
	\begin{pmatrix} 
		ie^{(1-i)t} & -ie^{(1+i)t}\\
e^{(1-i)t} & e^{(1+i)t}
\end{pmatrix} \cdot \int
%---
\begin{pmatrix}
	\frac1{2i} e^{(-1+i)t} & \frac12 e^{(-1+i)t}\\
	-\frac{1}{2i} e^{(-1-i)t} & -\frac12 e^{(-1-i)t}\\
\end{pmatrix} \begin{pmatrix} \sin(t) \\ \cos(t) \end{pmatrix} \id t \\
	&=
	\begin{pmatrix} 
		ie^{(1-i)t} & -ie^{(1+i)t}\\
e^{(1-i)t} & e^{(1+i)t}
\end{pmatrix} \cdot \int
%---
\frac12 \begin{pmatrix}
-i \sin(t)e^{(-1+i)t} +  \cos(t)e^{(-1+i)t}\\
	-i \sin(t)e^{(-1-i)t} - \cos(t)e^{(-1-i)t}
\end{pmatrix} \id t \\
	&=
	\begin{pmatrix} 
		ie^{(1-i)t} & -ie^{(1+i)t}\\
e^{(1-i)t} & e^{(1+i)t}
\end{pmatrix} \cdot \int
%---
\frac12 \begin{pmatrix}
	e^{(-1+i)t}(\cos(t) -i \sin(t))\\
	e^{(-1-i)t}(-\cos(t) -i\sin(t))\\
\end{pmatrix} \id t \\
	&=
	\begin{pmatrix} 
		ie^{(1-i)t} & -ie^{(1+i)t}\\
e^{(1-i)t} & e^{(1+i)t}
\end{pmatrix} \cdot \int
%---
\frac12 \begin{pmatrix}
	e^{-t+ti}e^{-it}\\
	-e^{-t-ti}e^{it}
\end{pmatrix} \id t \\
	&=
	\begin{pmatrix} 
		ie^{(1-i)t} & -ie^{(1+i)t}\\
e^{(1-i)t} & e^{(1+i)t}
\end{pmatrix} \cdot \int
%---
\frac12 \begin{pmatrix}
	e^{-t}\\
	-e^{-t}
\end{pmatrix} \id t \\
	&=
	\begin{pmatrix} 
		ie^{(1-i)t} & -ie^{(1+i)t}\\
e^{(1-i)t} & e^{(1+i)t}
\end{pmatrix} \cdot
%---
\frac12 \begin{pmatrix}
	-e^{-t}\\
	e^{-t}
\end{pmatrix} \\
&=\frac12\begin{pmatrix}
-ie^{-it} + ie^{it}\\
e^{-it} + e^{it}
\end{pmatrix} = \begin{pmatrix} -\sin(t) \\ \cos(t) \end{pmatrix} \\
\end{align*}


\item
	Allgemeine Inhomogene L"osung:
	\begin{align*}
		y(t) = y_H(t) + y_{SP}(t) = 
	c_1\begin{pmatrix}i\\1\end{pmatrix}e^{(1-i)t} + c_2\begin{pmatrix}-i\\1\end{pmatrix}e^{(1+i)t} + \begin{pmatrix} -\sin(t)\\ \cos(t) \end{pmatrix}
	\end{align*}

\end{itemize}


\section*{A35}
\begin{enumerate}
	\item
		Aus Gewohnheitsgr"unden: $x\rightarrow y $
		\begin{align*}
			y'' + \frac{\gamma}m y' = -g
		\end{align*}
		\begin{itemize}
			\item
				Homogene L"osung:
				\begin{align*}
					\lambda^2 + \frac{\gamma}m\lambda &\overset!= 0\\
					\lambda(\lambda+\frac{\gamma}m) &= 0\\
					&\Rightarrow \lambda_1 = 0,\, \lambda_2 = -\frac{\gamma}m
				\end{align*}
				\begin{align*}
					y_H = c_1 + c_2e^{-\frac{\gamma}m t}
				\end{align*}
			\item
				Spezielle L"osung:
				\begin{align*}
					y_{S} &= tA\\
					y'_S &= A\\
					y''_S &= 0\\
								&\Rightarrow \text{Einsetzen in inhomogene DGL}\\
						 &0 + \frac{\gamma}mA = -g\\
						 &\Rightarrow A = -\frac{gm}\gamma,\, y_S(t) = -\frac{gmt}\gamma  
				\end{align*}
			\item
				Allgemeine inhomogene L"osung:
				\begin{align*}
					y(t) = c_1 + c_2e^{-\frac\gamma{m}t} -\frac{gmt}\gamma
				\end{align*}
		\end{itemize}
	\item
		\begin{align*}
			y'(t) &= -\frac{\gamma}mc_2e^{-\frac\gamma{m}t} - \frac{gm}\gamma\\
			y(0) &= 2000\\
			y'(0) &= 0\\
			m &= 80\\
			g &= 3.71 \text{  (Mars)}\\
			\gamma &= \underbrace{-\frac12 \cdot c_W \cdot A \cdot \rho}_{\text{von \url{http://bit.ly/1V8IpAu}}} \\
			&\approx -\frac12 \cdot \underbrace{0.78}_{\text{stehender Mensch}} \cdot \underbrace{0.09}_{\text{ca. Fl"ache Mensch}} 
									 \cdot \underbrace{0.0069715}_{\text{Luftdichte auf dem Mars}} \\
									 & & \approx -0.000245
		\end{align*}
								 \item
		\begin{align*}
			c_1 + c_2 = 2000\\
			-\frac\gamma{m}c_2 - \frac{gm}\gamma = 0\\
			c_2 = -\frac{gm^2}{\gamma^2}\\
			\Rightarrow c_2 = -29054560599.750107\\
			\Rightarrow c_1 = -29054562599.750107
		\end{align*}
	\item
		\begin{align*}
			y'(t) = -\frac{\gamma}m c_2e^{-\frac\gamma{m}t} - \frac{g m}\gamma\\
			\lim_{t\rightarrow \infty} y'(t) = -\frac{gm}\gamma
		\end{align*}
		
	\item
		$\gamma$ h"angt ist proportional zur Gr"o"se des fallenden Objekts, je gr"o"ser der Fallschirm, desto gr"o"ser $\gamma$, desto kleiner der Betrag von $-\frac{gm}\gamma$

\end{enumerate}


\section*{A36}
		\begin{align*}
			y''' + a_2y'' + a_1y' + a_0y = f(t)
		\end{align*}
\begin{enumerate}
	\item
		\begin{align*}
			y(t) &= (c_1 + c_2t + c_3t^2)e^{-2t}\\
							&\Rightarrow \text{ Dreifacher Eigenwert } -2\\
						 &\Rightarrow(\lambda +2)^3 =  \lambda^3+6 \lambda^2+12 \lambda+8\\
						 &\Rightarrow a_2 = 6, a_1 = 12, a_0 = 8, f(t) = 0\\
					&\Rightarrow y''' + 6y'' + 12y' + 8y = 0
		\end{align*}

	\item
		\begin{align*}
			y(t) &=c_1te^t + c_2e^{7t} + c_3e^{-5t}\\
							&\Rightarrow \lambda_{1,2} = 1, \lambda_3 = 7, \lambda_4=-5\\
						 &\Rightarrow \text{ Nicht m"oglich 4 EW bei einer DGL 3. Grades zu bekommen}
		\end{align*}

	\item
		\begin{align*}
			y(t) &= c_1e^t + c_2\sin(t) + c_3 \cos(t) + t^4\\
							&\Rightarrow \lambda_1 = 1, \lambda_{2,3}= \pm i\\
						 &\Rightarrow (\lambda -1)(\lambda^2+1) = \lambda^3 -\lambda^2+\lambda -1\\
			y_H &= c_1e^t + c_2\sin(t) + c_3\cos(t)\\
			y_S &= t^4\\
			DGL_H &= y''' - y'' + y' - y = 0\\
						&a_2 = -1, a_1 = 1, a_2 = -1, f(t) =  -t^4 + 4 t^3 - 12 t^2 + 24 t
		\end{align*}

	\item
		\begin{align*}
			y(t) = c_1e^t + c_2\sin(t) + c_3\sin(2t) + t^4
		\end{align*}
		Nicht m"oglich, da man immer Paare von komplex-konjugierten Eigenwerten haben muss, woraus folgt, dass man ein Paar bestehend aus $e^{at}\cos(bt)$ und $e^{at}\sin(bt)$. Hier sind sowohl zwei mal Sinuse, als auch unterschiedliche $b$s.
\end{enumerate}
\end{document}
