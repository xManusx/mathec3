\documentclass[fleqn,12pt]{scrartcl}
\newcommand{\blattn}{Blatt 9}
\usepackage{cancel}
\usepackage[utf8]{inputenc}
\usepackage{color}
\usepackage[ngerman]{babel}
\usepackage{amssymb}
\usepackage{amsthm}
\usepackage{amsmath}
\usepackage{hyperref}
\usepackage{wasysym}
\usepackage{scrpage2}
\usepackage{tikz}
\usetikzlibrary{intersections}
\pagestyle{scrheadings}
\clearscrheadfoot
\ohead{\pagemark}
\ihead{Magnus Berendes, 21752155}
\ifoot{\today}
\ofoot{\blattn}
%\setheadtopline{1pt}
\setheadsepline{0.4pt}
\setfootsepline{0.4pt}
\usepackage{enumitem}
\setenumerate[0]{label=\alph*)}
\newcommand{\id}{\, \mathrm{d}}
\newcommand{\intl}{\int\displaylimits}
% New definition of square root:
% it renames \sqrt as \oldsqrt
\let\oldsqrt\sqrt
% it defines the new \sqrt in terms of the old one
\def\sqrt{\mathpalette\DHLhksqrt}
\def\DHLhksqrt#1#2{%
	\setbox0=\hbox{$#1\oldsqrt{#2\,}$}\dimen0=\ht0
	\advance\dimen0-0.2\ht0
	\setbox2=\hbox{\vrule height\ht0 depth -\dimen0}%
{\box0\lower0.4pt\box2}}

\newcommand{\karos}[2]{
	\begin{tikzpicture}
		\draw[step=0.5cm, color=gray] (0,0) grid (#1 cm , #2 cm);
	\end{tikzpicture}
}
\newcommand{\abs}[1]{
	\left \vert #1 \right \vert
}
\newcommand{\absbb}[1]{
	\left \Vert #1 \right \Vert
}

\begin{document}
\section*{\blattn}
Bearbeitet/Zur Korrektur:

\noindent
\begin{Form}
	\CheckBox{Aufgabe A25}\\
	\CheckBox{Aufgabe A26}\\
	\CheckBox{Aufgabe A27}\\
	%\CheckBox{Aufgabe A4}
\end{Form}

\section*{A25}
\begin{align*}
	y' &= \cos(t)(y+3\sqrt{y})\\
	\frac1{y+3\sqrt{y}} \id y &= \cos(t) \id t\\
	\int \frac1{y+3\sqrt{y}} \id y &= \int \cos(t) \id t && u := \sqrt{y}, \, \id y \rightarrow 2u \id u \\
	\int \frac{1}{u^2 + 3u}  2u \id u&= \sin(t)+C\\
	2 \int \frac{1}{u+4} \id u &= \sin(t)+C \\
	\ln(u+3) &= \frac12(\sin(t) +C) && u \rightarrow \sqrt{y}\\
	\ln(\sqrt{y} + 3) &= \frac12(\sin(t)+C)\\
	\sqrt{y} &= e^{\frac12(\sin(t)+C)} - 3\\
	y &= \left(e^{\frac12(\sin(t)+C)} -3 \right)^2
\end{align*}
Definitionsbereichbedingungen:
\begin{itemize}
	\item
		$\sqrt{y} \Rightarrow y\geq0$: erfüllt, da quadratische Funktion.
	\item
		$\frac1{\sqrt{y}} \Rightarrow y>0$
	\item
		$\sqrt{y} = e^{\frac12(\sin(t)+C)} - 3 \wedge y > 0 \Rightarrow e^{\frac12(\sin(t)+C)} \neq 3$
		\begin{align*}
			e^{\frac12(\sin(t)+C)} &> 3\\
			\frac12(\sin(t)+C) &> \ln 3\\
			\sin(t)&> 2\ln 3-C\\
			t &> \arcsin(2\ln 3 -C)\\
		\end{align*}
\end{itemize}
\begin{align*}
	\Rightarrow \mathbb{D} = \left(\arcsin(2\ln3-C), \infty\right)
\end{align*}
\section*{A26}
\begin{align*}
	y' &= -2ty + t^3,\, y(0) = 1\\
\end{align*}
\begin{itemize}
	\item
		Homogene L"osung:
		\begin{align*}
			y' &= -2ty \\
			\int \frac1y \id y &= \int -2t \id t \\
			\ln \abs{y} &= -t^2 + c \\
			y &= \pm e^{-t^2 + c} = \pm e^c \cdot e^{-t^2}\\
			y_\text{hom} &= Ce^{-t^2},\, C \in \mathbb{R}\\
		\end{align*}
	\item
		V.d.K.:
		\begin{align*}
			C&\rightarrow C(t)\\
			y(t) &\overset!= C(t) \cdot e^{-t^2}\\
			y'(t) &= C'(t)\cdot e^{-t^2} + C(t)\cdot e^{-t^2}\cdot (-2t)\\
			C'(t)\cdot e^{-t^2} + \cancel{(\dots)} &= \cancel{(\dots)}+ t^3 \\
			C'(t) &= \frac{t^3}{e^{-t^2}} = e^{t^2}t^3\\
			C(t) &= \int e^{t^2}t^3 \id t && u := t^2, \, \id y \rightarrow \frac1{2\sqrt{u}} \id u\\
			C(t) &= \int e^uu\sqrt{u}  \frac1{2\sqrt{u}} \id u\\
					 &= \frac12 \int e^uu \id u\\
					 &= \frac12 (e^u u  - \int e^u \id u) = \frac12(e^uu - e^u)\\
					 &= \frac12e^u(u-1) && u \rightarrow t^2\\
					 &= \frac12e^{t^2}(t^2 -1) \\
			y_\text{sp} &= \frac12e^{t^2}(t^2 -1)\cdot e^{-t^2} \\
													 &= \frac12(t^2 -1)
		\end{align*}
	\item
		Allgemeine L"osung:
		\begin{align*}
			y(t) &= y_\text{hom}(t) + y_\text{sp}(t)\\
							&= Ce^{-t^2} +\frac12(t^2 -1)\\
		\end{align*}

	\item
		Anfangswert:
		\begin{align*}
			y(0) &= 1\\
			1 &= Ce^{0} +\frac12(-1)\\
			\frac32 &= C\\
			y(t) &= \frac32e^{-t^2} +\frac12(t^2 -1)\\
							 &= \frac12(3e^{-t^2} + t^2 - 1)
		\end{align*}
	\item
		Definitionsbereicheinschränkungen:
		\begin{itemize}
			%\item $\frac1{2\sqrt{u}} \Rightarrow u \neq 0 \Rightarrow t \neq 0$
			\item Keine: \\
				$\mathbb{D} = (-\infty, \infty)$
	\end{itemize}


\end{itemize}
\section*{A27}
\begin{enumerate}
	\item
\begin{align*}
	y' &= \frac{t^2 + y^2}{t\sqrt{1-t^2}}+\frac{y}t, \, y(\frac12) = \frac{\sqrt{3}}{6}\\
			 &\qquad z:=\frac{y}{t} \Rightarrow y = zt,\, y' = z't + z,\, z\left(\frac12\right) = 2\cdot \frac{\sqrt{3}}6 = \frac1{\sqrt{3}}\\
	y' &= \frac{t^3 + y^2t}{t^2\sqrt{1-t^2}} + \frac{y}{t} = \frac{t^3}{t^2\sqrt{1-t^2}} + \frac{y^2t}{t^2\sqrt{1-t^2}} + \frac{y}{t}\\
	z't + z &=  \frac{t}{\sqrt{1-t^2}} + \frac{z^2 t}{\sqrt{1-t^2}} + z\\
	z' &= \frac{z^2 + 1}{\sqrt{1-t^2}}\\
	\frac1{z^2 + 1} \id z &= \frac1{\sqrt{1-t^2}} \id t\\
	\intl_\frac{1}{\sqrt{3}}^z \frac{1}{\tilde z^2 + 1} \id \tilde z &= \intl_{\frac1z}^t \frac1{\sqrt{1-\tilde t^2}} \id \tilde t\\
	\arctan\left(z\right) - \arctan \left(\frac1{\sqrt3}\right) &= \arcsin\left(t\right) - \arcsin\left(\frac12\right)\\
	\arctan(z) &= \arcsin(t)\\
	z &= \tan\left(\arcsin(t)\right)\\
	y &= \tan(\arcsin(t))\cdot t
\end{align*}
Definitionsmenge:
\begin{itemize}
	\item Wegen Gleichung aus Angabe: $t \neq 0 \wedge t \neq 1$
	\item Wegen $\arcsin(t)$: $-1\leq t \leq 1$
	\item Wegen $\tan(\arcsin(t))$: $\arcsin(t) \neq \pm \frac\pi2 \Rightarrow t \neq \pm 1$
	\item Wegen Startwert:  $t>0$
\end{itemize}
\begin{align*}
	\mathbb{D} = \left(0, 1\right)
\end{align*}

\item
	\begin{align*}
		y(t) &= \tan(\arcsin(t))\cdot t = \frac{\sin(\arcsin(t))}{\cos(\arcsin(t))} \cdot t\\
						 &= \frac{t^2}{\cos(\arcsin(t))} \\
						 &= \frac{t^2}{\pm \sqrt{1-\sin^2(\arcsin(t))}}\\
						 &= \frac{t^2}{\sqrt{1-t^2}}
	\end{align*}
	Das $\pm$ kann weggelassen werden, da $\arcsin$ auf $\mathbb{D}$ von $(0, 1) \rightarrow \left(0, \frac\pi2\right)$ abbildet und $\cos$ von $\left(0, \frac\pi2\right) \rightarrow (1, 0)$ abbildet und $\cos(\arcsin(t))$ somit positiv ist.

	\end{enumerate}

\end{document}
