\documentclass[fleqn,12pt]{scrartcl}
\usepackage[utf8]{inputenc}
\usepackage{color}
\usepackage[ngerman]{babel}
\usepackage{amssymb}
\usepackage{amsthm}
\usepackage{amsmath}
\usepackage{gauss}
\usepackage{braket}
\usepackage{hyperref}
\usepackage{wasysym}
\usepackage{scrpage2}
\usepackage{tikz}
\usetikzlibrary{intersections}
\pagestyle{scrheadings}
\clearscrheadfoot
\ohead{\pagemark}
\ihead{Magnus Berendes, matrikelnr}
\ifoot{\today}
\ofoot{\blattn}
%\setheadtopline{1pt}
\setheadsepline{0.4pt}
\setfootsepline{0.4pt}
\usepackage{enumitem}
\setenumerate[0]{label=\alph*)}
\newcommand{\id}{\, \mathrm{d}}
\newcommand{\intl}{\int\displaylimits}
% New definition of square root:
% it renames \sqrt as \oldsqrt
\let\oldsqrt\sqrt
% it defines the new \sqrt in terms of the old one
\def\sqrt{\mathpalette\DHLhksqrt}
\def\DHLhksqrt#1#2{%
	\setbox0=\hbox{$#1\oldsqrt{#2\,}$}\dimen0=\ht0
	\advance\dimen0-0.2\ht0
	\setbox2=\hbox{\vrule height\ht0 depth -\dimen0}%
{\box0\lower0.4pt\box2}}

\newcommand{\karos}[2]{
	\begin{tikzpicture}
		\draw[step=0.5cm, color=gray] (0,0) grid (#1 cm , #2 cm);
	\end{tikzpicture}
}
\newcommand{\abs}[1]{
	\left \vert #1 \right \vert
}
\newcommand{\absbb}[1]{
	\left \Vert #1 \right \Vert
}

%TODO
\newcommand{\blattn}{Blatt 10}
\begin{document}
\section*{\blattn}
Bearbeitet/Zur Korrektur:

\noindent
\begin{Form}
	\CheckBox{Aufgabe A28}\\
	\CheckBox{Aufgabe A29}\\
	\CheckBox{Aufgabe A30}\\
	%\CheckBox{Aufgabe A4}
\end{Form}

\section*{A28}
\begin{enumerate}
	\item
\begin{align*}
	y' &= \abs{y}^{\frac23},\, y(0) = 0\\
	f(x)&:= \abs{x}^{\frac23}\\
				&f(x) \text{ stetig in } \mathbb{R} \Rightarrow \exists \text{ eine lokale L"osung}
\end{align*}
\item
	Vorraussetzung f"ur Satz: $f(x)$ Lipschitz-stetig:
	\begin{align*}
		\abs{f(x) - f(y)} &\leq L\cdot \abs{x-y},\, L \in \mathbb{R}
	\end{align*}
	\begin{align*}
																	 x > 0 :&
		f(x) = x^{\frac23} \\
		&\Rightarrow f'(x) = \frac2{3\oldsqrt[3]{x}} \text{ ist lokal beschr"ankt}\\
											  & \Rightarrow
		\text{ L existiert lokal } \Rightarrow f \text{ lokal L-stetig}\\
		x < 0 :&
		\text{ analog zu } x > 0 \text{, da }\abs{ \cdot }\\
		x = 0 :&
		\lim_{x\rightarrow +0} f'(x) = \lim_{x \rightarrow +0} \frac2{3\oldsqrt[3]{x}} = \infty \text{ und damit nicht beschr"ankt}\\
		  & \Rightarrow \text{ nicht L-stetig}
	\end{align*}
	Damit kann der Satz nicht angewendet werden, garantiert uns also nichts.
\item
	$y(t) = 0$
\item
	\begin{align*}
		y'&=\abs{y}^{\frac23}\\
		\frac{\id y}{\id t} &= \abs{y}^{\frac23}\\
		\int \frac1{\abs{y}^{\frac23}} \id y &= \int 1 \id t\\
		\int {y}^{-\frac23} \id y &= t\\
																						 &\text{(da uns nur L"osungen $y\geq 0$ interessieren,}\\
																&\text{k"onnen Betragsstriche gl"ucklicherweise vernachl"assigt werden)}\\
		3 y^{\frac13} &= t\\
		3{\oldsqrt[3]{y}} &= t\\
		y &= \frac{t^3}{27} + C\\
		y(0) &= 0 \Rightarrow \, C=0
	\end{align*}
\item
	\begin{align*}
		y' &= \abs{y}^{\frac23}\\
	\frac1{9}t^2 &= \abs{\frac{t^3}{27}}^{\frac23}\\
		\frac1{9}t^2 &= \abs{\frac{t}{3}}^2 = \frac19 t^2\\
		y(0) &= \frac0{27} = 0
	\end{align*}
\item
	Im Fall $y(t) = \frac{t^3}{27}$ nimmt die Masse des Regentropfens - in Abh"angigkeit der Zeit - kubisch zu. Im Fall $y(t) = 0$ gibt es keinen Staubpartikel an dem sich ein neuer Tropfen bilden k"onnte

	\section*{A29}
	\begin{enumerate}
		\item
			\begin{align*}
			y' = My,\, y(0) = \begin{pmatrix} 3\\3\\3\end{pmatrix},\, M = \begin{pmatrix}
				1 & 2 & -1\\
				2 & 4 & -2 \\
			-1 & -2 & 1 \end{pmatrix}
			\end{align*}
			Berechnung der Eigenwerte:
			\begin{align*}
				&\det\begin{pmatrix}
					1-\lambda & 2 & -1\\
					2 & 4 -\lambda & -2\\
					-1 & -2 & 1-\lambda
				\end{pmatrix} \overset!= 0\\
				0 &= (1-\lambda)^2(4-\lambda)+8-(4-\lambda)-8(1-\lambda)\\
					&= \lambda^3 + 6\lambda^2\\
				 &= \lambda^2(6-\lambda)\\
				 &\Rightarrow \lambda_1 = 0,\, \lambda_2 = 6
			\end{align*}
			Berechnung der dazugeh"origen Eigenr"aume:
			\begin{align*}
				&\lambda = 0:\\
				&\begin{gmatrix}[p]
					1&2 & -1\\
					2 & 4 & -2\\
					-1 & -2 & 1
					\rowops
					\add[-\frac12]{0}{1}
					\add{0}{2}
				\end{gmatrix} = \begin{pmatrix} 1 & 2 & -1 \\ 0 & 0 & 0 \\ 0 & 0 &0\end{pmatrix}\\
																					 &\Rightarrow x_1 = x_3 -2x_2\\
			\end{align*}
			\begin{align*}
			ER_0 = \Set{ c_1 \begin{pmatrix} -2\\1\\0\end{pmatrix} + c_2 \begin{pmatrix} 1 \\ 0 \\ 1\end{pmatrix} |  c_1, c_2 \in \mathbb{R}}
			\end{align*}
			\begin{align*}
			&\lambda = 6:\\
			 &\begin{gmatrix}[p]
					-5 & 2 & -1\\
					2 & -2 & -2 \\
					-1 & -2 & -5
					\rowops
					\add{2}{0}
					\add{2}{1}
				\end{gmatrix} = \begin{gmatrix}[p]
					6 & 0 & 6 \\
				0 & 1 & 2\\
					1 & 2 & 5 
					\rowops
					\mult{0}{\cdot \frac16}
					\add[-\frac16]{0}{2}
				\end{gmatrix} =\\
				&	= \begin{gmatrix}[p]
					1 & 0 & 1\\
					0 & 1 & 2\\
					0 & 2 & 4
					\rowops
					\add[-2]{1}{2}
				\end{gmatrix} = \begin{pmatrix} 1 & 0 & 1\\ 0 & 1 & 2\\ 0 & 0 & 0\end{pmatrix}\\
																					 &\Rightarrow x_1 = -x_3,\, x_2 = -2x_3\\
			\end{align*}
			\begin{align*}
			ER_6 = \Set{c_1 \begin{pmatrix}-1\\-2\\1\end{pmatrix} | c_1 \in \mathbb{R}}
			\end{align*}
			\begin{align*}
			y(t) = c_1 \begin{pmatrix} -1 \\ -2 \\ 1 \end{pmatrix}\cdot e^{6t} + c_2\begin{pmatrix}-2 \\ 1 \\ 0 \end{pmatrix} + c_3 \cdot \begin{pmatrix} 1 \\ 0 \\ 1 \end{pmatrix}
			\end{align*}

	\end{enumerate}


\end{enumerate}

\end{document}
