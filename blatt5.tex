\documentclass[fleqn,12pt]{scrartcl}
\usepackage[utf8]{inputenc}
\usepackage{color}
\usepackage[ngerman]{babel}
\usepackage{amssymb}
\usepackage{amsthm}
\usepackage{amsmath}
\usepackage{hyperref}
\usepackage{wasysym}
\usepackage{scrpage2}
\pagestyle{scrheadings}
\clearscrheadfoot
\ohead{\pagemark}
\ihead{Magnus Berendes, matrikelnr}
\ifoot{\today}
\ofoot{\blattn}
%\setheadtopline{1pt}
\setheadsepline{0.4pt}
\setfootsepline{0.4pt}
\usepackage{enumitem}
\setenumerate[0]{label=\alph*)}
\newcommand{\id}{\, \mathrm{d}}
\newcommand{\intl}{\int\displaylimits}
% New definition of square root:
% it renames \sqrt as \oldsqrt
\let\oldsqrt\sqrt
% it defines the new \sqrt in terms of the old one
\def\sqrt{\mathpalette\DHLhksqrt}
\def\DHLhksqrt#1#2{%
	\setbox0=\hbox{$#1\oldsqrt{#2\,}$}\dimen0=\ht0
	\advance\dimen0-0.2\ht0
	\setbox2=\hbox{\vrule height\ht0 depth -\dimen0}%
{\box0\lower0.4pt\box2}}


\newcommand{\blattn}{Blatt 5}
\begin{document}
\section*{\blattn}
Bearbeitet/Zur Korrektur:

\noindent
\begin{Form}
	\CheckBox{Aufgabe A13}\\
	\CheckBox{Aufgabe A14}\\
	\CheckBox{Aufgabe A15}\\
	%\CheckBox{Aufgabe A4}
\end{Form}

\section*{A13}
\begin{enumerate}
	\item
		\begin{align*}
			\Phi(x) &= \frac12 \cos^3 x\\
			\Phi'(x) &= \frac32 \cos^2(x) \cdot (-\sin x) \cdot 1\\
											 &= (\frac32 - \frac32\sin^2(x)) \cdot (-\sin x)\\
											 &=\frac32 \sin^3(x) - \frac32 \sin x && y:=\sin x\\
											 &=f(y) = \frac32 y^3 -\frac32 y,\, y\in[-1, 1]\\
			f'(y) &= \frac92 y^2 - \frac32 \overset!= 0\\
								 &\Rightarrow \sin x = \pm \frac1{\sqrt{3}}\\
								 &\Rightarrow x_{1,2} = \pm \frac1{\sqrt{3}}\\
			\vert \Phi'(x_1) \vert &\approx \vert 0.577 \cdot (-1) \vert = 0.577\\
			\vert \Phi'(x_2) \vert &\approx 0.577\\
													&\text{Grenzen des Definitionsbereichs einsetzen:}\\
			y_3 &= -1,\, y_4 = 1\\
			\sin(x_3) &= -1 \Rightarrow x_3 = -\frac\pi2\\
			\sin(x_4) &= 1 \Rightarrow x_4 = \frac\pi2\\
			\vert \Phi'(x_3) \vert &= 0\\
			\vert \Phi'(x_4) \vert &= 0\\
																									 &\Rightarrow \sup_{x \in [-1,1]} \vert \Phi'(x) \vert = 0.577 < 1\\
																									&\Rightarrow \Phi(x) \text{ kontrahierend}
		\end{align*}
	\item Da $\Phi(x)$ kontrahierend ist, gibt es genau einen Fixpunkt $x_*$ für den $x_*=\Phi(x_*)$ gilt. Somit hat $x = \frac12 \cos^3 x$ nur eine Lösung.
\item
				\begin{align*}
					x_1&=0.5\\
					x_2&=0.33793561091735264\\
					x_3&=0.4198695547443537\\
					x_4&=0.38070198010453765\\
					x_5&=0.40011151692990193\\
					x_6&=0.39063732555067615\\
					x_7&=0.3952987779161624\\
					x_8&=0.39301388768981166\\
					x_9&=0.3941359725607877\\
					x_10&=0.3935854326150147\\
					x_11&=0.3938556714333518\\
					x_12&=0.39372305090763887\\
					x_13&=0.39378814189563477\\
					x_14&=0.3937561965292067\\
					x_15&=0.39377187509123035\\
\end{align*}
Ab $x_8$ ändert sich die zweite Nachkommastelle beim Runden nicht mehr.
\begin{align*}
	\Rightarrow x_* \approx 0.39
\end{align*}

\end{enumerate}

\section*{A14}
\begin{enumerate}
	\item
		\begin{align*}
			\Phi(x)& = \frac{x}2 + \frac1x\\
			%\Phi(x)& \text{ kontrahierend auf } [1,2] \Leftrightarrow \Vert \Phi(x) - \Phi(y) \Vert \leq k \cdot \Vert x - y \Vert, \, &x,y \in [1,2], \\ & &k \in (0,1)
			\Phi(x)& \text{ kontrahierend auf } [1,2] \Leftrightarrow \sup_{x\in[1,2]} \Vert \Phi'(x) \Vert < 1\\
			\Phi'(x)& = \frac12 - \frac1{x^2}  \\
			\Phi''(x)& = \frac2{x^3}\not = 0 \,\forall x \in [1,2] \Rightarrow \Phi'(x) \text{ keine lokalen Extrema} \\
			\Phi'(1)& = -\frac12 \\
			\Phi'(2)& = \frac14\\
			k =& \sup_{x\in[1,2]} \Vert \Phi'(x) \Vert = \frac12< 1\\
		\end{align*}
		$k$ ist die Kontraktionskonstante
		\item
			\begin{align*}
				\Phi'(x) &= \frac12 - \frac1{x^2} \overset!= 0\\
												 &\Rightarrow x_{1,2} = \pm \sqrt{2}, \, -\sqrt{2} \not \in [1,2]\\
				\Phi(\sqrt{2})&\approx 1.41\\
				\Phi(1) &= 1.5 \\
				\Phi(2) &= 1.5 \\
			\end{align*}
			Da weder am Extremum, noch an den Rändern das Intervall $[1,2]$ verlassen wird, ist klar, dass $\Phi(x)$ eine Abbildung von $[1,2] \rightarrow [1,2]$ und somit eine Selbstabbildung ist.
		\item
			\begin{align*}
				x_1&=1.5 \\
x_2&=1.4166666666666665\\
x_3&=1.4142156862745097\\
x_4&=1.4142135623746899\\
			\end{align*}

		\item
			\begin{align*}
				&\Vert x_4 - x_* \Vert \leq \frac{k}{1-k} \Vert x_4 - x_3 \Vert \\
				\Rightarrow& \vert x_4 - x_* \vert \leq \vert x_4 -x_3 \vert \cdot \frac{0.5}{0.5}\\
				\Rightarrow& \vert x_4 - x_* \vert \leq 12 \cdot 10^{-6}\\
				\Rightarrow& \text{ Der Abstand zwischen } x_* \text{ und } x_4 \text{ ist höchstens } 2.12\cdot 10^{-6}
			\end{align*}


\end{enumerate}
\section*{A15}
\begin{enumerate}
	\item

		\begin{align*}
			\Phi(x) &= x - \frac{f(x)}{f'(x)}\\
			\Phi &: \mathbb{R} \rightarrow \mathbb{R},\, \mathbb{R} \text{ ist Banachraum}
		\end{align*}
		Um die Kontraktion von $\Phi(x)$ zu zeigen muss gelten: $\sup_{x\in\mathbb{R}} \vert \Phi'(x) \vert < 1$
		\begin{align*}
			\Phi'(x) = 1 - \frac{f'(x) \cdot f'(x) - f(x) \cdot f''(x)}{f'(x) \cdot f'(x)}
			= \frac{f(x) \cdot f''(x)}{f'(x)^2}\\
			\text{ Aus der Vorraussetzung folgt: } \frac{\vert f(x) \cdot f''(x)\vert}{\vert f'(x) \vert^2} < k, \, k \in (0,1)\\
			\Rightarrow \left \vert \frac{f(x) \cdot f''(x)}{f'(x)^2} \right \vert < k\\
			\Rightarrow \sup_{x \in \mathbb{R}} \vert \Phi'(x) \vert < k\, \Rightarrow \, \Phi(x)\, \text{kontrahierend}
		\end{align*}
		Damit gilt der Fixpunktsatz für $\Phi(x) = x - \frac{f(x)}{f'(x)}$, das Newton-Verfahren konvergiert also unabhängig vom Startwert, falls die Vorbedingung $\vert f(x) f''(x) \vert < k \vert f'(x) \vert^2\quad \forall x \in \mathbb{R}$ gilt
		\item
			\begin{enumerate}[label=\roman*)]
					\item
						\begin{align*}
							f(x) = \sinh x,\, f'(x) = \cosh x, \, f''(x) = \sinh x
						\end{align*}
						\begin{align*}
							\vert \sinh^2 x \vert &< k \cosh^2 x\\
							\left \vert \frac{e^x - e^{-x}}{2} \right \vert ^2 &< k \left( \frac{e^x + e^{-x}}{2} \right)^2\\
							\left \vert \frac{e^x - e^{-x}}{2} \right \vert &< \sqrt{k}\frac{e^x + e^{-x}}{2}\\
							\vert e^x - e^{-x} \vert &< \sqrt{k}(e^x + e^{-x})\\
							\frac{\overbrace{\vert e^x - e^{-x} \vert}^{<e^x}}{\underbrace{e^x + e^{-x}}_{>e^x}} &< \sqrt{k}\\
					&\Rightarrow \frac{{\vert e^x - e^{-x} \vert}}{{e^x + e^{-x}}} < 1\\
							\exists k \in (0,1): 
							&\Rightarrow \frac{{\vert e^x - e^{-x} \vert}}{{e^x + e^{-x}}} < \sqrt{k}\\
						\end{align*}
					\item
						\begin{align*}
							f(x) = 2e^{3x}, \, f'(x) = 6e^{3x},\, f''(x) = 18e^{3x}
						\end{align*}
						\begin{align*}
						\end{align*}
							\begin{align*}
								\vert \underbrace{2e^{3x} \cdot 18 e^{3x}}_{>0} \vert &< k\cdot (6e^{3x})^2, k \in (0,1)\\
								36 e^{6x} &\not< k \cdot 36e^{6x}\\
								&\Rightarrow (3) \text{ gilt nicht}
							\end{align*}
						\item
							\begin{align*}
								&f(x) = 3x + 4, \, f'(x) = 3,\, f''(x) = 0\\
							\end{align*}
							\begin{align*}
								\vert (3x + 4)\cdot 0 \vert &< k \cdot \vert 3 \vert^2, k \in (0,1)\\
								0 &< k\cdot 9\\
									 &\Rightarrow \text{Gilt für alle } k\in (0,1)
							\end{align*}
							\item
								\begin{align*}
									f(x) = x^2, \, f'(x) = 2x, \, f''(x) = 2
								\end{align*}
								\begin{align*}
									\vert 2x^2 \vert &< k \cdot \vert (2x)^2 \vert, k \in (0,1)\\
									2x^2 &< k \cdot 4x^2\\
									&\Rightarrow \text{Gilt für } k > \frac12
								\end{align*}
				\end{enumerate}
\end{enumerate}

\end{document}
