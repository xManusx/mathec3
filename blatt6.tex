\documentclass[fleqn,12pt]{scrartcl}
\usepackage[utf8]{inputenc}
\usepackage{color}
\usepackage[ngerman]{babel}
\usepackage{amssymb}
\usepackage{amsthm}
\usepackage{amsmath}
\usepackage{hyperref}
\usepackage{wasysym}
\usepackage{scrpage2}
\pagestyle{scrheadings}
\clearscrheadfoot
\ohead{\pagemark}
\ihead{Magnus Berendes, matrikelnr}
\ifoot{\today}
\ofoot{\blattn}
%\setheadtopline{1pt}
\setheadsepline{0.4pt}
\setfootsepline{0.4pt}
\usepackage{enumitem}
\setenumerate[0]{label=\alph*)}
\newcommand{\id}{\, \mathrm{d}}
\newcommand{\intl}{\int\displaylimits}
% New definition of square root:
% it renames \sqrt as \oldsqrt
\let\oldsqrt\sqrt
% it defines the new \sqrt in terms of the old one
\def\sqrt{\mathpalette\DHLhksqrt}
\def\DHLhksqrt#1#2{%
	\setbox0=\hbox{$#1\oldsqrt{#2\,}$}\dimen0=\ht0
	\advance\dimen0-0.2\ht0
	\setbox2=\hbox{\vrule height\ht0 depth -\dimen0}%
{\box0\lower0.4pt\box2}}

\usepackage{tikz}
\newcommand{\karos}[2]{
	\begin{tikzpicture}
		\draw[step=0.5cm, color=gray] (0,0) grid (#1 cm , #2 cm);
	\end{tikzpicture}
}
\usetikzlibrary{intersections}
\newcommand{\abs}[1]{
	\left \vert #1 \right \vert
}
\newcommand{\absbb}[1]{
	\left \Vert #1 \right \Vert
}

%TODO hier blatt!
\newcommand{\blattn}{Blatt 6}
\setcounter{MaxMatrixCols}{20}
\begin{document}
\section*{\blattn}
Bearbeitet/Zur Korrektur:

\noindent
\begin{Form}
	\CheckBox{Aufgabe A16}\\
	\CheckBox{Aufgabe A17}\\
	\CheckBox{Aufgabe A18}\\
	%\CheckBox{Aufgabe A4}
\end{Form}

\section*{A16}
\begin{enumerate}
	\item
		$A$ ist diagonaldominant:
		\begin{align*}
			\abs{4} &> \abs{2} + \abs{1}\\
			\abs{4} &> \abs{1} + \abs{1}\\
			\abs{4} &> \abs{0} + \abs{3}\\
										&\Rightarrow \text{ Jacobi-Verfahren konvergiert}
		\end{align*}
	\item

		\begin{align*}
			&x_0: [0, 0, 0]^T\\
	 &x_1: 
[ 1.75 , 1.5  , 1.75]^T\\
&x_2: 
[ 0.5625,  0.625  , 0.625 ]^T\\
&x_3: 
[ 1.28125 ,  1.203125 , 1.28125 ]^T\\
		\end{align*}
	\item
		\begin{align*}
			k &\geq \frac{\absbb{x_n - x_*}}{\absbb{x_{n-1} - x_*}}\\
		\end{align*}
		Fehler f"ur $x_1 = 0.6666667 \quad \leftarrow $ Gr"o"ster Fehler $\Rightarrow 0.666667 \leq k \leq 1$ \\
		Fehler f"ur $x_2 = 0.59375$\\
		Fehler f"ur $x_3 = 0.6447368$\\

\end{enumerate}
\section*{A17}

\begin{enumerate}
	\item
		\begin{tikzpicture}[x=.5cm, y=.5cm,domain=-9:9,smooth]
			%Raster zeichnen
			\draw [color=gray!50]  [step=5mm] (-11,-10) grid (11,10);
			% Achsen zeichnen
			\draw[->,thick] (-10,0) -- (10,0) node[right] {$x$};
			\draw[->,thick] (0,-9) -- (0,9) node[above] {$y$};
			% Achsen beschriften
			\foreach \c in {-8,-6,...,-2,2,4,...,8}{
				\draw (\c,-.1) -- (\c,.1) node[below=4pt] {$\scriptstyle\c$};
				\draw (-.1,\c) -- (.1,\c) node[left=4pt] {$\scriptstyle\c$};
			}
			\node[below left]{$\scriptstyle0$};
			%Funktionen zeichnen:
			\path[name path=rahmen,clip](-11,-9) rectangle (11,9);
			\draw[name path=p1] plot (\x, {(-(2/5)*\x)});  
			\draw[color=gray!80,name path=p2] plot (\x, {1 - (1/2)*\x});  
			%\draw[color=gray!80,name path=p3] plot (\x, {\x-3});  
		\end{tikzpicture}
	\item
		\begin{tikzpicture}[x=.5cm, y=.5cm,domain=-9:9,smooth]
			%Raster zeichnen
			\draw [color=gray!50]  [step=5mm] (-11,-10) grid (11,10);
			% Achsen zeichnen
			\draw[->,thick] (-10,0) -- (10,0) node[right] {$x$};
			\draw[->,thick] (0,-9) -- (0,9) node[above] {$y$};
			% Achsen beschriften
			\foreach \c in {-8,-6,...,-2,2,4,...,8}{
				\draw (\c,-.1) -- (\c,.1) node[below=4pt] {$\scriptstyle\c$};
				\draw (-.1,\c) -- (.1,\c) node[left=4pt] {$\scriptstyle\c$};
			}
			\node[below left]{$\scriptstyle0$};
			%Funktionen zeichnen:
			\path[name path=rahmen,clip](-11,-9) rectangle (11,9);
			\draw[name path=p1] plot (\x, {(-(2/5)*\x)});  
			\draw[color=gray!80,name path=p2] plot (\x, {1 - (1/3)*\x});  
			\draw[color=gray!80,name path=p3] plot (\x, {\x-3});  
		\end{tikzpicture}
	\item
		\begin{tikzpicture}[x=.5cm, y=.5cm,domain=-9:9,smooth]
			%Raster zeichnen
			\draw [color=gray!50]  [step=5mm] (-11,-10) grid (11,10);
			% Achsen zeichnen
			\draw[->,thick] (-10,0) -- (10,0) node[right] {$x$};
			\draw[->,thick] (0,-9) -- (0,9) node[above] {$y$};
			% Achsen beschriften
			\foreach \c in {-8,-6,...,-2,2,4,...,8}{
				\draw (\c,-.1) -- (\c,.1) node[below=4pt] {$\scriptstyle\c$};
				\draw (-.1,\c) -- (.1,\c) node[left=4pt] {$\scriptstyle\c$};
			}
			\node[below left]{$\scriptstyle0$};
			%Funktionen zeichnen:
			\path[name path=rahmen,clip](-11,-9) rectangle (11,9);
			\draw[name path=p1] plot (\x, {(-(2/5)*\x)});  
			\draw[color=gray!80,name path=p2] plot (\x, {\x});  
			\draw[color=gray!80,name path=p3] plot (\x, {1-\x});  
			%\draw[color=gray!80,name path=p4] plot (\y, {-2});  
		\end{tikzpicture}
\end{enumerate}
\section*{A18}

%\begin{align*}
	%&\max \{ c^T x \;|\; A x \leq b, x \geq 0 \} \\
		%c^T &= [2, -3, 0, 0, 1]^T\\
	%A &= \begin{pmatrix}
	%1 && 2 && 0 && 0 && 0 \\
	%3 && 6 && 0 && 0 && 0 \\
	%0 && 0 && 1 && 4 && 3 \\
	%0 && 0 && 0 && 2 && 3 \\
	%0 && 0 && 2 && 0 && 3 \\
	%1 && 3 && 0 && 0 && -4\\
	%0 && 1 && 0 && 1 && 0 \\
	%4 && 2 && -2 && 0 && 0 \\
	%0 && 0 && -1 && 0 && 0 \\
%\end{pmatrix}\\
%b &= \begin{pmatrix}
%18\\
%54\\
%120\\
%50\\
%190\\
%1\\
%20\\
%-30\\
%10
%\end{pmatrix}
%\end{align*}

\begin{align*}
	-\min\, &-2x_1 +3x_2 - x_5 \\
	s.t.\, &x_1 + 2x_2 = 18\\
	&3x_1 + 6x_2 = 54\\
	&x_3 + x_4 + 3x_5 = 120 \\
	&2x_4 + 3x_5 = 50\\
	&2x_3 + 3x_5 = 190 \\
	&x_1 + 3x_2 - 4x_5 + x_6 = 1 \\
	&x_2 + x_4 + x_7 = 20\\
	&4x_1 + 2x_2 - 2x_3 + x_8 = -30\\
	&-x_3 + x_9 = 10\\
	&x_1,x_2,x_5, x_6, x_7, x_8, x_9 \geq 0\\
	&x_3 = x_3^+ - x_3^-\\
	&x_4 = x_4^+ - x_4^-\\
	&x_3^+, x_3^-, x_4^+, x_4^- \geq 0
\end{align*}

\begin{align*}
	A =& 
	\begin{pmatrix}
		x_1 & x_2 & x_3^+ & x_3^- & x_4^+ & x_4^- & x_5 & x_6 & x_7 & x_8 & x_9\\
		1 & 2 & 0 & 0 & 0 & 0 & 0 & 0 & 0 & 0 & 0\\
		3 & 6 & 0 & 0 & 0 & 0 & 0 & 0 & 0 & 0 & 0\\
		0 & 0 & 1 & 1 & 1 & 1 & 3 & 0 & 0 & 0 & 0\\
		0 & 0 & 0 & 0 & 2 & 2 & 3 & 0 & 0 & 0 & 0\\
		0 & 0 & 2 & 2 & 0 & 0 & 3 & 0 & 0 & 0 & 0\\
		1 & 3 & 0 & 0 & 0 & 0 &-4 & 1 & 0 & 0 & 0\\
		0 & 1 & 0 & 0 & 1 & 1 & 0 & 0 & 1 & 0 & 0\\
		4 & 2 &-2 &-2 & 0 & 0 & 0 & 0 & 0 & 1 & 0\\
		0 & 0 &-1 &-1 & 0 & 0 & 0 & 0 & 0 & 0 & 1\\
	\end{pmatrix}\\
	\vec b =& 
	\begin{pmatrix}
		18 \\
		54\\
		120\\
		50\\
		190\\
		1\\
		20\\
		-30\\
		10
	\end{pmatrix}, \,
	\vec c =
	\begin{pmatrix}
		-2 \\ 3 \\ 0 \\ 0 \\ -1 \\ 0 \\ 0 \\ 0 \\ 0 \\ 0 \\ 0
	\end{pmatrix}
\end{align*}
:


\end{document}
