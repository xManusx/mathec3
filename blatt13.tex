\documentclass[fleqn,12pt]{scrartcl}
\usepackage[utf8]{inputenc}
\usepackage{color}
\usepackage[ngerman]{babel}
\usepackage{amssymb}
\usepackage{amsthm}
\usepackage{amsmath}
\usepackage{gauss}
\usepackage{braket}
\usepackage{hyperref}
\usepackage{wasysym}
\usepackage{scrpage2}
\usepackage{tikz}
\usetikzlibrary{intersections}
\pagestyle{scrheadings}
\clearscrheadfoot
\ohead{\pagemark}
\ihead{Magnus Berendes, matrikelnr}
\ifoot{\today}
\ofoot{\blattn}
%\setheadtopline{1pt}
\setheadsepline{0.4pt}
\setfootsepline{0.4pt}
\usepackage{enumitem}
\setenumerate[0]{label=\alph*)}
\newcommand{\id}{\, \mathrm{d}}
\newcommand{\intl}{\int\displaylimits}
% New definition of square root:
% it renames \sqrt as \oldsqrt
\let\oldsqrt\sqrt
% it defines the new \sqrt in terms of the old one
\def\sqrt{\mathpalette\DHLhksqrt}
\def\DHLhksqrt#1#2{%
	\setbox0=\hbox{$#1\oldsqrt{#2\,}$}\dimen0=\ht0
	\advance\dimen0-0.2\ht0
	\setbox2=\hbox{\vrule height\ht0 depth -\dimen0}%
{\box0\lower0.4pt\box2}}

\newcommand{\karos}[2]{
	\begin{tikzpicture}
		\draw[step=0.5cm, color=gray] (0,0) grid (#1 cm , #2 cm);
	\end{tikzpicture}
}
\newcommand{\abs}[1]{
	\left \vert #1 \right \vert
}
\newcommand{\absbb}[1]{
	\left \Vert #1 \right \Vert
}

%TODO
\newcommand{\blattn}{Blatt 13}
\begin{document}
\section*{\blattn}
Bearbeitet/Zur Korrektur:

\noindent
\begin{Form}
	\CheckBox{Aufgabe A37}\\
	\CheckBox{Aufgabe A38}\\
	\CheckBox{Aufgabe A39}\\
	%\CheckBox{Aufgabe A4}
\end{Form}


Anmerkung zur Notation:
\begin{align*}
	[a]_n = [b]_n \Leftrightarrow a \equiv b \mod n
\end{align*}
\section*{A37}
\begin{enumerate}
	\item
		\begin{align*}
			&n \equiv 0 \mod 11 \Leftrightarrow Q_{alt}(n) \equiv 0  \mod 11\\
		\left[Q_{alt}(n)\right]_{11} &= \left[\sum_{i=0}^k (-1)^in_i\right]_{11} = \sum_{i=0}^k [-1]_{11}^i [n_i]_{11}
																							 = \sum_{i=0}^k
																							 [10]_{11}^i [n_i]_{11} = [n]_{11}\\
																							 &\Rightarrow n \equiv Q_{alt}(n) \mod 11\\
		&\Rightarrow Q_{alt}(n) \equiv 0 \mod 11\Rightarrow n \equiv 0 \mod 11 
		\end{align*}


	\item
		\begin{align*}
			\left[7^{77}\right]_{10} &= [7]_{10}^{77}\\ 
																											 &= [-3]^{77}_{10}\\
															 &=
			[-1]_{10}^{77}[3]^{77}_{10}\\
			&=	[-1]_{10} [3]_{10}^{64}[3]_{10}^{13} = [-1]_{10} [3]_{10}^{2^6} [3]_{10}^3 [3]_{10}^{2^5}\\
			&= [9]_{10} [9]_{10}^{6} [7]_{10} [9]_{10}^5\\
			&= [3]_{10} [-1]_{10}^6 [9]^2_{10} [9]^3_{10}\\
			&= [3]_{10} [1]_{10} [1]_{10}  [9]_{10}\\
			&= [7]_{10}
		\end{align*}
		\begin{align*}
			7^{77} \mod 10 = 7
		\end{align*}

\end{enumerate}


\section*{A38}
\begin{enumerate}
	\item
		\begin{align*}
			612 &= 6 \cdot 96 + 36\\
			96 &= 2 \cdot 36 + 24 \\
			36 &= 1 \cdot 24 + 12\\
			24 &= 2 \cdot 12 + 0
		\end{align*}
		\begin{align*}
			\gcd(96, 612) = 12
		\end{align*}
		
	\item
		\begin{align*}
			x^3 -4x &= \left(x+4\right)\left(x^2 -4x+4\right) + \left(8x -16\right)\\
			\left(x^2 - 4x + 4\right) &= \left(\frac18 x - \frac14\right) \left(8x - 16\right) + 0 \\
		\end{align*}
		\begin{align*}
			\gcd(x^3 -4x, x^2 -4x + 4) = 8x - 16 = 8(x - 2)
		\end{align*}
\end{enumerate}


\section*{A39}
\begin{enumerate}
	\item
		\begin{align*}
			21 &= 3 \cdot 7 \\
			\left(\mathbb{Z}_{21}^*, \cdot \right) &= \Set{[i]_{21}| i\in \Set{1,2,4,5,8,10,11,13,16,17,19,20}   }
		\end{align*}

	\item
		\begin{align*}
			[8]_{21}^{-1}
			[10]_{21}^{-1}
			[17]_{21}^{-1}
			\text{ existieren, }
			[9]_{21}^{-1}
			[14]_{21}^{-1}
			\text{ nicht}
		\end{align*}
		
	\item
		\begin{align*}
			21 &= 2 \cdot 8 + 5 \\
			8 &= 1 \cdot 5 + 3 \\
			5 &= 1 \cdot 3 + 2 \\
			3 &= 1 \cdot 2 + 1 \\
			2 &= 2 \cdot 1 + 0 \\
			\Rightarrow 1 &= 3 - 1 \cdot 2\\
				&= 3 - 1 \cdot (5 - 1 \cdot 3)\\
			 &= 2 \cdot 3 - 1 \cdot  5 \\ 
			 &= 2 \cdot (8 - 1 \cdot 5) - 1 \cdot 5 \\
			 &= 2 \cdot 8 - 3 \cdot 5\\
			 &= 2 \cdot 8 - 3 \cdot (21 - 2 \cdot 8 )\\
			 &= 8 \cdot 8 - 3 \cdot 21\\
			&\Rightarrow [8]_{21}^{-1} = [8]_{21}
		\end{align*}
		\begin{align*}
			21 &= 2 \cdot 10 + 1\\
			10 &= 10 \cdot 1 + 0\\
			\Rightarrow 1 &= 21 - 2 \cdot 10 \\
																 &\Rightarrow [10]_{21}^{-1} = [-2]_{21} = [19]_{21}
		\end{align*}
		\begin{align*}
			21 &= 1 \cdot 17 + 4\\
			17 &= 4 \cdot 4 + 1 \\
			4 &= 4 \cdot 1 + 0\\
			\Rightarrow 1 &= 17 - 4\cdot 4\\
																&=17 - 4 \cdot (21 - 1 \cdot 17) \\
									 &= 5 \cdot 17 - 4 \cdot 21\\
									 &\Rightarrow [17]_{21}^{-1} = [5]_{21}
		\end{align*}
\end{enumerate}
\end{document}
