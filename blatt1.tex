\documentclass[fleqn,13pt]{scrartcl}
\usepackage[utf8]{inputenc}
\usepackage{color}
\usepackage[ngerman]{babel}
\usepackage{amssymb}
\usepackage{amsthm}
\usepackage[fleqn]{amsmath}
\usepackage{nccmath}
\usepackage{hyperref}
\usepackage{scrpage2}
\usepackage{enumitem}
\pagestyle{scrheadings}
\clearscrheadfoot
\ohead{\pagemark}
\ihead{Magnus Berendes, matrikelnr}
\ifoot{\today}
\ofoot{Blatt 1}
%\setheadtopline{1pt}

\setenumerate[0]{label=\alph*)}
\setheadsepline{0.4pt}

\begin{document}
\section*{Blatt 1}
\subsection*{Aufgabe A1}

\begin{enumerate}
	\item
		\begin{equation*}
			g(x,y,z) = x^2 + y^3 + z^4 
		\end{equation*}

		\begin{equation*}
			\nabla_g(x,y,z) =
			\begin{pmatrix}
				2x \\
				3y^2 \\
				4z^3
			\end{pmatrix}
		\end{equation*}
		\begin{equation*}
			\mathcal{H}_g(x,y,z) =
			\begin{pmatrix}
				2 & 0 & 0 \\
				0 & 6y & 0 \\
				0 & 0 & 12z^2
			\end{pmatrix}
		\end{equation*}
		\begin{equation*}
			\mathcal{H}_g(3,3,-4) =
			\begin{pmatrix}
				2 & 0 & 0 \\
				0 & 18 & 0 \\
				0 & 0 & 184
			\end{pmatrix}
		\end{equation*}

		Bei einer Diagnoalmatrix gilt: $\text{EW}_i = a_{i,i}$, die Eigenwerte sind also $2, 18, 184$

		$\Rightarrow \mathcal{H}_g(3,3,-4)$ positiv definit

	\item

		\begin{equation*}
			h(x,y) = xy\cos(y)
		\end{equation*}
		\begin{equation*}
			\nabla_h(x,y) =
			\begin{pmatrix}
				y\cos(y)\\
				x\cos(y) - xy\sin(y)\\
			\end{pmatrix}
			=
			\begin{pmatrix}
				y\cos(y)\\
				x(\cos(y) - y\sin(y))\\
			\end{pmatrix}
		\end{equation*}
		\begin{equation*}
			\mathcal{H}_h(x,y) = 
			\begin{pmatrix}
				0 & \cos(y) - y\sin(y)\\
				\cos(y) - y\sin(y) & -x(2\sin(y) + y\cos(y))
			\end{pmatrix}
		\end{equation*}
		\begin{equation*}
			\mathcal{H}_h(1, \frac\pi2) = 
			\begin{pmatrix}
				0 & -\frac\pi2\\
				-\frac\pi2 & -2
			\end{pmatrix}
		\end{equation*}
		Charakteristisches Polynom von $\mathcal{H}_h(1, \frac\pi2) = -\lambda \cdot (-2 - \lambda) - (\frac\pi2)^2 = \lambda^2 +2\lambda - \frac{\pi^2}4$
		\begin{equation*}
			\Rightarrow \lambda_{1,2} = \frac{-2 \pm \sqrt{4 + \pi^2}}2 = -1 \pm \sqrt{1 + \frac{\pi^2}4}
		\end{equation*}

		\begin{equation*}
			\sqrt{1 + \frac{\pi^2}4} > 1 \Rightarrow (\lambda_1 > 0, \lambda_2 < 0) \Rightarrow \text{Die Matrix ist indefinit}
		\end{equation*}

\end{enumerate}

\subsection*{Aufgabe A2}
\begin{enumerate}
	\item
		\begin{equation*}
			f(x,y) = \frac52x^2 + y^2 -3xy + x - y
		\end{equation*}
		\begin{equation*}
			\nabla_f(x,y) = 
			\begin{pmatrix}
				5x - 3y + 1\\
				2y - 3x - 1
			\end{pmatrix} \overset{!}{=} 0
		\end{equation*}
	$\Rightarrow x = 1; y = 2$, $\begin{pmatrix}1 \\2\end{pmatrix}$ station"arer Punkt von $f(x,y)$

	\item
		\begin{equation*}
			\mathcal{H}_f(x,y) =
			\begin{pmatrix}
				5 & -3 \\
				-3 & 2
			\end{pmatrix}
		\end{equation*}
		Charakteristisches Polynom von $\mathcal{H}_f(1,2) = (5-\lambda)(2-\lambda)-9 = \lambda^2 - 7\lambda + 1 \overset{!}{=} 0$
		\begin{equation*}
			\lambda_{1,2} = \frac{7 \pm \sqrt{45}}2; \lambda_1 = 6.85\text{\dots} > 0; \lambda_2 = 0.15\text{\dots} > 0
		\end{equation*}
	$\Rightarrow$ Alle EW $>0$ $\Rightarrow$ Matrix positiv definit $\Rightarrow$ $\begin{pmatrix}1\\2\end{pmatrix}$ Minimum


\end{enumerate}

\subsection*{Aufgabe A3}

\begin{enumerate}

	\item
		\begin{equation*}
			f(x,y) = \ln(x) + y
		\end{equation*}
		\begin{equation*}
			y = -\frac{x^4}4
		\end{equation*}
		($x^4 > 0 \Rightarrow y < 0$)
		\begin{equation*}
			\tilde f(x) = \ln(x) - \frac{x^4}4
		\end{equation*}
		\begin{equation*}
			\tilde f'(x) = \frac1x - x^3 = \frac1x(1 - x^4)\overset!= 0 \Rightarrow x_{1,2} = \pm1; \text{ ($x_2$ f"allt weg, da $x \overset!> 0$)}
		\end{equation*}
		\begin{equation*}
			\tilde f''(x) = -x^{-2} -3x^2 = -\frac1{x^2} - 3x^2
		\end{equation*}
		\begin{equation*}
			\tilde f''(1) = -4, \tilde f(1) = -\frac14
		\end{equation*}

	$\Rightarrow \tilde f(x)$ ist im Punkt $\begin{pmatrix}1\\ -\frac14\end{pmatrix}$ rechtsgekr"ummt, $\begin{pmatrix}1\\ -\frac14\end{pmatrix}$ ist also ein Maximum von $\tilde f(x)$ und damit ein Maximum von $f(x,y)$ unter der Nebenbedingung $y + \frac{x^4}4 = 0$



\end{enumerate}

\end{document}

