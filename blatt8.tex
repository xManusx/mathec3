\documentclass[fleqn,12pt]{scrartcl}
\usepackage[utf8]{inputenc}
\usepackage{color}
\usepackage[ngerman]{babel}
\usepackage{amssymb}
\usepackage{amsthm}
\usepackage{amsmath}
\usepackage{hyperref}
\usepackage{wasysym}
\usepackage{scrpage2}
\usepackage{tikz}
\usepackage{pgfplots}
\usetikzlibrary{intersections}
\pagestyle{scrheadings}
\clearscrheadfoot
\ohead{\pagemark}
\ihead{Magnus Berendes, 21752155}
\ifoot{\today}
\ofoot{\blattn}
%\setheadtopline{1pt}
\setheadsepline{0.4pt}
\setfootsepline{0.4pt}
\usepackage{enumitem}
\setenumerate[0]{label=\alph*)}
\newcommand{\id}{\, \mathrm{d}}
\newcommand{\intl}{\int\displaylimits}
% New definition of square root:
% it renames \sqrt as \oldsqrt
\let\oldsqrt\sqrt
% it defines the new \sqrt in terms of the old one
\def\sqrt{\mathpalette\DHLhksqrt}
\def\DHLhksqrt#1#2{%
	\setbox0=\hbox{$#1\oldsqrt{#2\,}$}\dimen0=\ht0
	\advance\dimen0-0.2\ht0
	\setbox2=\hbox{\vrule height\ht0 depth -\dimen0}%
{\box0\lower0.4pt\box2}}

\newcommand{\karos}[2]{
	\begin{tikzpicture}
		\draw[step=0.5cm, color=gray] (0,0) grid (#1 cm , #2 cm);
	\end{tikzpicture}
}
\newcommand{\abs}[1]{
	\left \vert #1 \right \vert
}
\newcommand{\absbb}[1]{
	\left \Vert #1 \right \Vert
}
\newcommand{\abl}[2]{
	\frac{\id #1}{\id #2}
}

\newcommand{\blattn}{Blatt 8}
\begin{document}
\section*{\blattn}
Bearbeitet/Zur Korrektur:

\noindent
\begin{Form}
	\CheckBox{Aufgabe A22}\\
	\CheckBox{Aufgabe A23}\\
	\CheckBox{Aufgabe A24}\\
	%\CheckBox{Aufgabe A4}
\end{Form}

\section{A22}
\begin{align*}
	mx'' + kx = 0
\end{align*}
\begin{enumerate}
	\item
		Z.z.: $E = E_{spann} + E_{kin} = \frac12kx^2 + \frac12m(x')^2$ konstant "uber t $\Rightarrow \frac{\id E}{\id t} = 0$
		\begin{align*}
			\frac{\id E}{\id t} &= kx\cdot x' + mx' \cdot x''\\
																						 &= x'\cdot \underbrace{(kx + mx'')}_{=0} = 0\\
																		& && && \Box
		\end{align*}
	\item
			\begin{enumerate}[label=\roman*)]
				\item $x = x_0\cos(\omega t)$
					\begin{align*}
						\Rightarrow x''= -x_0\omega^2 \cos(\omega t)&\\
						-mx_0\omega^2 \cos(\omega t) + kx_0\cos(\omega t) &= 0\\
						%x_0\cos(\omega t)(k - m\omega^2) & = 0\\
						m\cdot\left(-x_0\omega^2 \cdot \cos(\omega t) \right) + kx_0\cos(\omega t) &= 0
																						 %&\Rightarrow x_0 = 0\\
					\end{align*}
					\begin{align*}
						\text{Wenn } t \neq 0:&\\
						&\omega = \frac{\pi n}t - \frac{\pi}{2t}, \, n \in \mathbb{Z}\\
						\text{Wenn } m > 0:& \text{ (wie in Angabe)}\\
						&\omega = \pm\sqrt{\frac{k}m} \wedge x_0 = 0\\
						\text{Wenn } \cos(\omega t) = 0&, \, \text{also } \omega t \in \frac\pi2 + n, \, n \in \mathbb{Z}\\
					\end{align*}
				\item
					$x= x_0 \cosh  (\omega t), \, x'' = x_0 \omega^2 \cosh(\omega t)$
					\begin{align*}
						0 &= mx_0 \omega^2 \cosh(\omega t) + kx_0 \cosh(\omega t)\\
							 &= x_0 \cosh(\omega t)(m\omega^2 + k)\\
							 && \Rightarrow x_0 = 0, \omega \text{ beliebig}\\
					\end{align*}
				\item
					$x = x_0e^{\omega t}, \, x'' = x_0 \omega^2 e^{\omega t}$
					\begin{align*}
						0 &\overset!= x_0 e^{\omega t}(m\omega^2 + k)\\
						&& x_0 = 0 , \omega \text{ beliebig}\\
					\end{align*}
			\end{enumerate}

\end{enumerate}
\section{A23}
\begin{enumerate}
	\item
		Siehe Teilaufgabe b)

	\item
\begin{enumerate}[label=\arabic*.]
	\item
		\begin{align*}
			\abl yt &= yt\\
			\frac1y \id y &= t \id t\\
			\int \frac1y \id y &= \int t \id t\\
			\ln y &= \frac12 t^2 + C\\
			y &= e^{\frac{t^2}2 + C}\\
			y &\overset!> 0 \\
			\Rightarrow \mathbb{D}_y &= (-\infty, \infty)
		\end{align*}
		\begin{align*}
			y(0) &= 1, t = 0&\\
			1 &= e^{0 + C}\\
			&\Rightarrow C = 0
		\end{align*}
	\item
		\begin{align*}
			\abl yt &= \sqrt{y}\\
			\frac1{\sqrt{y}} \id y &= 1 \id t\\
			\int y^{-\frac12} \id y &= \int 1 \id t\\
			2 \sqrt{y} &= t + C\\
			\sqrt{y} &= \frac{t+C}{2}\\
			y &= \frac{(t+C)^2}4\\
			y \overset!\neq 0 &\Rightarrow (t+C) \neq 0\\
			\mathbb{D}_1 &= (-\infty, -C), \, \mathbb{D}_2 = (-C, \infty)\\
									 &t = 1, y = 1\\
			1 &= \frac{(1 + C)^2}4\\
			C &= 1 \vee C = -3\\
			\mathbb{D}_1 &= (-1, \infty), \mathbb{D}_2 = (3, \infty)
		\end{align*}

	\item
		\begin{align*}
			\abl yt &= \cos (yt) \\
		\end{align*}
		Nicht l"osbar mit T.d.V., da Funktion  nicht von der Form $y' = f(y) \cdot g(t)$

	\item
		\begin{align*}
			\abl yt &= e^{-y}\\
			\abl yt &= \frac{1}{e^{y}}\\
			e^y \id y &= 1 \id t\\
			\int e^y \id y &= \int 1 \id t\\
			e^y &= t+C \\
			y &= \ln (t+C) \\
			\mathbb{D}_1 &= (-C, \infty)\\
			y &= 1, t = e\\
			1 &= \ln (e + C)\\
			e &= e + C\\
				&\Rightarrow C=0\\
			\mathbb{D} &= (0, \infty)
		\end{align*}
		
	\item
		\begin{align*}
			\abl yt &= y + t\\
		\end{align*}
		Nicht mit T.d.V. l"osbar, Funktion nicht von der Form $y' = f(y) \cdot g(t)$
\end{enumerate}

\end{enumerate}
\section{A24}
\begin{enumerate}
	\item
		$y' = (1-y)y$: Nicht lineare, explizite, skalare, autonome Differentialgleichung 1. Ordnung
	\item
		$y$ streng monoton wachsend $\Leftrightarrow y' > 0$
		\begin{align*}
			y' = \underbrace{(1-y)}_{\geq 0}\underbrace{y}_{\geq 0}
		\end{align*}

		F"ur $y\neq 0$ und $y\neq 1$ ist $y' > 0$. Wenn die Populationsgr"o"se 0 ist, kann sie nicht wachsen, wenn sie die Obergrenze erreicht hat, auch nicht mehr.
		
		Genau genommen ist $y$ somit nur \emph{monoton wachsend}
	\item
		\begin{align*}
			\abl yt &= y - y^2\\
			\frac1{y-y^2} \id y &= 1 \id t\\
			\int \frac1{y-y^2} \id y &= \int 1 \id t\\
			\int \frac1y + \frac1{1-y} \id y &= t && \text{ (linke Seite Partialbruchzerlegung)}\\
			\int \frac1y \id  y + \int \frac1{1-y} \id y &= t + C\\
			\ln y - \ln(1-y) &= t + C\\
			e^{\ln y - \ln(1-y)} &= e^{t+C}\\
			\frac{e^{\ln y}}{e^{\ln (1-y)}} &= e^{t+C}\\
			\frac{y}{1-y} &= e^{t+C}\\
			%(y &= e^{t+C} - e^{t+C}y)\\
			\frac{1-y}{y} &= \frac1{e^{t+C}}\\
			\frac1y - 1 &= \frac1{e^{t+C}}\\
			\frac1y &= \frac1{e^{t+C}} + 1\\
			\frac1y &= \frac{1+e^{t+C}}{e^{t+C}}\\
			y &= \frac{e^{t+C}}{1+e^{t+C}}\\
			&y\in (0, 1)\\
			\mathbb{D}_y &= (-\infty, \infty)\\
			y(0) &= \frac12\\
			\frac12 &= \frac{e^C}{1+e^C} \\
			2 &= \frac1{e^c} + 1\\
			e^C &= 1\\
					&\Rightarrow C=0\\
		\end{align*}
		\begin{align*}
			y = \frac{e^t}{1+e^t}, \, \mathbb{D} = (-\infty, \infty)
		\end{align*}
	\item \quad \\
		%\begin{tikzpicture}
			%\begin{axis}[
					%samples=60,
					%domain=0:100, xmax=100,
					%restrict y to domain=0:1,
					%axis lines=left,
					%%y=0.5cm/3,
					%x=0.5cm,
					%grid=both,
					%xtick={0,...,10},
					%ytick={0,0.1,...,0.9,1},
					%%compat=newest,
					%xlabel=$x$, xlabel style={at={(1,0)}, anchor=west},
					%ylabel=$y$, ylabel style={rotate=-90,at={(0,1)}, anchor=south}
				%]
				%\addplot [red] {x};
				%\addplot [black] {x^2};
			%\end{axis}
		%\end{tikzpicture}
		\begin{tikzpicture} 
			\begin{axis}[
					samples=50,
					height=15cm,
					width=15cm,
					grid=major,
					domain=0:10
				] 
				\addplot[mark=none, ultra thick] {(e^x)/(1+e^x)} 
				node [right, below]{$y(x)$};
			\end{axis} 
		\end{tikzpicture}
		Am Anfang findet starkes, schnelles Wachstum statt. Je weiter die Kurve sich an 100\% ann"ahert, desto langsamer wird das Wachstum, bis es sich an die horizontale Asymptote $y=1$ anschmiegt: die Populationsgr"o"se ist (fast) 100\% der Obergrenze.
\end{enumerate}

\end{document}
